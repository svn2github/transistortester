\chapter{Der optiboot Bootloader für AVR Mikrocontroller}

\section*{}
Der optiboot Bootloader wurde in der Sprache C von Peter Knight und
Bill Westfield entwickelt. Die Version 6.2 habe ich als Basis
für die hier beschriebene überarbeitete Assembler Version benutzt.
Dabei möchte ich betonen, daß ich den optiboot Bootloader nicht
neu erfunden habe, sondern lediglich weiter optimiert. 
Viele Anpassungen an verschiedene Zielprozessoren und spezielle
Platinenentwürfe waren bereits in der Version 6.2 vorhanden.
Es werden Teile des STK500 Kommunikations-Protokols benutzt,
die in der AVR061~\cite{stk500} von Atmel veröffentlicht wurde.


\section{Änderungen und Weiterentwicklung von Version 6.2}
Im wesentlichen habe ich das komplette Programm in die Assembler-
Sprache umgeschrieben und die Makefile so angepaßt, daß die Programmlänge
automatisch weiterverarbeitet wird und damit die Startadresse
des Bootloaders sowie die Fuses des ATmega richtig eingestellt werden.
Die eingeschlagene Lösung erzeugt während der Abarbeitung der Einzelschritte
für die Erzeugung des Programmcodes für den Bootloader noch weitere
kleine Dateien, welche in den nachfolgenden Schritten für die Anpassung
der Start-Adresse und der Fuses erforderlich sind.
Die Startadresse für den jeweiligen Zielprozessor ist abhängig von
der vorhandenen Flash-Speichergröße,
dem Speicherbedarf für den aktuellen Bootloader-Code und
der Kachelgröße, die für Bootloader beim Zielprozessor zur Verfügung steht.
Als Kachelgröße bezeichne ich die kleinste Speichergröße für Bootloader,
die der jeweilige Prozessor zur Verfügung stellen kann.


Bei Prozessoren wie der ATtiny84, die keine Bootloader Startadresse einstellen können,
wird die Seitengröße des Flash-Speichers für die Berechnung benutzt.
Beim ATtiny84 sind das 64 Bytes. Damit liegt die Startadresse des Bootloaders immer
am Anfang einer Flash Speicherseite. 

Bei allen anderen Zielprozessoren kann der Bootloader-Bereich mit den
Fuse-Bits BOOTSZ1 und BOOTSZ0 eingestellt werden (jweils mit den Werten 0 und 1).
Wenn man die beiden Bits zusammensetzt, ergibt sich daraus die
Bootloader-Größe BOOTSZ mit Werten zwischen 0 und 3.
Dabei bedeutet 3 immer den kleinsten mögliche Bootloader Speicherbereich.
Der Wert 2 gibt den doppelten, der Wert 1 den vierfachen und
der Wert 0 den acht-fachen Speicherbereich vor.
Die Tabelle~\ref{tab:bootsz} auf Seite~\pageref{tab:bootsz} gibt einen
Überblick für verschiedene Zielprozessoren.

\section{Automatische Größenanpassung in der optiboot Makefile}

Die Bootloader Startadresse und die benötigte Bootloadergröße wird
automatisch in der Makefile angepaßt. Für die Berechnungen werden
einige Zwischendateien erzeugt, was nur zusammen
mit den folgenden Linux Werkzeugen funktioniert:
\begin{description}
\item [bc] ein einfacher Rechner, der die Eingabe- und Ausgabe-Werte
sowohl dezimal als auch sedezimal (hex) verarbeiten kann.
\item [cat] gibt den Inhalt von Dateien auf der Standard-Ausgabe aus.
\item [cut] kann Teile von Zeilen einer Textdatei ausschneiden.
\item [echo] gibt den angegebenen Text auf der Standard-Ausgabe aus.
\item [grep] gibt nur Zeilen einer Textdatei mit dem angegebenen Suchtext aus.
\item [tr] kann Text-Zeichen ersetzen oder löschen.
\end{description}

Bisher ist die Funktion der Makefile nur mit einem Linux-System getestet.
Wahrscheinlich ist die Benutzung unter Windows nur möglich,
wenn das Cygwin Paket installiert wird.

Um die erzeugten Zwischendateien braucht man sich im Regelfall nicht zu kümmern. Hier
möchte ich aber wenigstend die Namen und die Bedeutung erwähnen:
\begin{description}
\item [BootPages.dat] enthält die Zahl der vom Bootloader benötigten Seiten.
Bei Prozessoren mit Bootloader Unterstützung kann die Zahl nur 1, 2, 4 oder 8 sein und
gibt an, das wievielfache der Mindest-Bootloadergröße verwendet wird.
Bei der virtuellen Bootloader Seite kann die Zahl beliebig sein und gibt die Zahl der
benötigten Flash-Speicherseiten an.
\item [BOOTSZ.dat] enthält eine Zahl zwischen 0 und 3 für die Einstellung der BOOTSZ0 und BOOTSZ1 Bits.
\item [BL\_StartAdr.dat] enthält die Startadresse des Bootloaders im sedezimal (hex) Format.
Die Startadresse wird berechnet aus der Flash-Speichergröße des jeweiligen Zielprozessors und
aus der Zahl der benötigten Speicherseiten.
\item [EFUSE.dat] enthält im sedezimal Format der Wert für die efuse. Die Makefile entscheidet
 abhängig vom Zielprozessor, ob diese Datei verwendet wird.
\item [HFUSE.dat] enthält im sedezimal Format der Wert für die hfuse. Die Makefile entscheidet
 abhängig vom Zielprozessor, ob diese Datei verwendet wird.
\end{description}

\section{Zielvorgaben für optiboot Makefile}

Die Steuerung des Ablaufs für die Generierung der Programmdaten aus
dem Quellcode ist in der Makefile festgelegt.
Außer der  Haupt Makefile gibt es noch weitere drei Erweiterungen
der Makefile, die automatisch von der Haupt Makefile integriert werden:
Makefile.1284, Makefile.atmel, und Makefile.extras .
Dabei können verschiedene Konfigurationen auch für einen Prozessortyp
festgelegt worden sein. In der Tabelle~\ref{tab:processors} sind die derzeit vordefinierten
Konfigurationen für AVR Prozessoren angegeben. Im Prinzip ist diese Liste natürlich erweiterbar.
Die einstellbaren Parameter sind aber auch in der Aufrufzeile des
make Programms als Parameter oder auch als Umgebungsvariable der Shell
einstellbar.

\begin{table}[H]
  \begin{center}
    \begin{tabular}{| c | c | c | c | c | c | c | c | c |}
    \hline
             Name  & MCU & AVR\_ & total & Flash & BP\_ & LFUSE & HFUSE & EFUSE  \\
                   &     & FREQ  & Flash & page  & LEN  &       &       &        \\
                   &     &       &  size & size  &      &       &       &        \\
    \hline
    \hline
         attiny84 & t84   & 16M? &  8K   &  64   & (64) &  62   &  DF   & FE \\
    \hline
         atmega8  & m8    & 16M  &  8K   &  64   & 256  &  BF   &  CC   &  - \\
    \hline
         atmega88 & m88   & 16M  &  8K   &  64   & 256  &  FF   &  DD   &  04 \\
    \hline
       atmega16   & m16   & 16M  &  16K  & 128   & 256  &  FF   &  9C   &  - \\
    \hline
       atmega168  &  m168  & 16M  &  16K & 128   & 256  &  FC   &  DD   &  04 \\
       atmega168p &  m168p & 16M  &  16K & 128   & 256  &  FC   &  DD   &  04 \\
    \hline
       atmega32   &  m32   & 16M  &  16K & 128   & 256  &  BF   &  CE   &  - \\
    \hline
       atmega328  &  m328  & 16M  &  32K & 128   & 512  &  FF   &  DE   &  05 \\
       atmega328p & m328p & 16M  &  32K  & 128   & 512  &  FF   &  DE   &  05 \\
    \hline
       atmega644p & m644p & 16M  &  64K  & 256   & 512  &  F7   &  DE   &  05 \\
    \hline
     atmega1284  & m1284  & 16M  & 128K  & 256   & 512  &  F7   &  DE   &  05 \\
     atmega1284p & m1284p & 16M  & 128K  & 256   & 512  &  F7   &  DE   &  05 \\
    \hline
     atmega1280  & m1280  & 16M  &  128K & 256   & 1K   &  FF   &  DE   &  05 \\
    \hline
    \end{tabular}
  \end{center}
  \caption{Prozessor targets für optiboot Makefile}
  \label{tab:processors}
\end{table}

Alle Angaben für Größen sind in Bytes angegeben, die Werte für die Fuses sind die sedezimalen
Werte (hex). Die Frequenzwerte werden in Hz angegeben, 16M entspricht also 16000000 Hz.
Die Standard Baud-Rate für die serielle Schnittstelle beträgt immer 115200.

Neben den universellen Konfigurationen gibt es auch Konfigurationen für bestimmte
Platinen oder Arbeitsumgebungen.
Die Tabelle~\ref{tab:boards} zeigt die unterschiedlichen Einstellungen.

\begin{table}[H]
  \begin{center}
    \begin{tabular}{| c | c | c | c | c | c | c | c | c | c | }
    \hline
             Name  & MCU & AVR\_ & BP\_ & L     & H     & E     & BAUD\_ & LED & SOFT\_ \\
                   &     & FREQ  & LEN  &  FUSE & FUSE  &  FUSE & RATE &     & UART \\
    \hline
    \hline
       luminet     & t84 &  1M   & 64v &  F7   &  DD   &  04   & 9600 & 0x  &  -   \\

    \hline
         virboot8  & m8    & 16M & 64v &       &       &       &      &     &      \\
    \hline
       diecimila  &  m168 & (16M) &     &  F7   &  DD   &  04   &      & 3x  &  -   \\
       lilypad    &  m168  & 8M  &      &  E2   &  DD   &  04   &  -   & 3x  &  -   \\
       pro8       &  m168  & 16M &      &  F7   &  C6   &  04   &  -   & 3x  &  -   \\
       pro16      &  m168  & 16M &      &  F7   &  DD   &  04   &  -   & 3x  &  -   \\
       pro20      &  m168  & 16M &      &  F7   &  DC   &  04   &  -   & 3x  &  -   \\
    atmega168p\_lp&  m168  & 16M &      &  FF   &  DD   &  04   &  -   &     &  -   \\
   xplained168pb  &  m168  &(16M)&      &       &       &      & 57600 &     &      \\
    \hline
       virboot328  & m328p & 16M & 128v &       &       &       &      &     &  -   \\
    atmega328\_pro8& m328p & 8M  &      &  FF   &  DE   &  05   &  -   & 3x  &  -   \\
   xplained328pb  &  m168  &(16M)&      &       &       &      & 57600 &     &      \\
   xplained328p   &  m168  &(16M)&      &       &       &      & 57600 &     &      \\
    \hline
        wildfire  & m1284p & 16M &      &       &       &      &   -   & 3xB5 &      \\
    \hline
       mega1280    & m1280 & 16M &      &  FF   &  DE   &  05   &  -   &     &  -   \\
    \hline
    \end{tabular}
  \end{center}
  \caption{vorkonfigurierte targets für optiboot Makefile}
  \label{tab:boards}
\end{table}

\section{Die Optionen für die optiboot Makefile}

Mit den Optionen wird die Eigenschaft des optiboot Bootloaders eingestellt.
Beispielsweise kann mit der Option SOFT\_UART veranlasst werden, daß ein
Softwareprogramm für die serielle Kommunikation verwendet werden soll.
Sonst wird, wenn vorhanden, die auf dem Chip integrierte  serielle Schnittstelle
mit den Pins TX (Transmit = senden) und RX (Receive = empfangen) benutzt.
Bei mehreren integrierten seriellen Schnittstellen wird normalerweise die erste
Schnittstelle mit den Nummer 0 verwendet. Es kann aber auch jede andere vorhandene
Schnittstelle mit der Option UART vorgegeben werden (UART=1 für die zweite Schnittstelle).
Bei der Hardware UART Schnittstelle sind die Pins für Empfangen (RX) und Senden (TX)
fest verknüpft. Bei der Software-Lösung für die serielle Schnittstelle sind alle Pins
des AVR Prozessors frei für die serielle Kommunikation wählbar. Die einzige Bedingung
ist, daß die Pins für digitale Eingabe (RX) beziehungsweise Ausgabe (TX) geeignet sind.
Näheres zu den möglichen Optionen findet man in der Übersicht~\ref{tab:options1}
und \ref{tab:options2}

\begin{table}[H]
  \begin{center}
    \begin{tabular}{| c | c | l |}
    \hline
   Name der        & Beispiel       & Funktion                                            \\
    Option         &                &                                                     \\
    \hline
    \hline
    F\_CPU         & F\_CPU=8000000 & Teilt dem Programm die Taktrate des Prozessors mit. \\
                   &                & Die Angabe erfolgt in Hz (Schwingungen pro Sekunde. \\
                   &                & Das Beispiel gibt eine Frequenz von 8 MHz an. \\
    \hline
    BAUD\_RATE     & BAUD\_         & Gibt die Baud-Rate für die serielle Kommunikation an. \\
                   &  RATE=9600      & Es werden immer 8 Datenbits ohne Parity verwendet. \\
    \hline
    SOFT\_UART     & SOFT\_UART=1   & Wählt für die serielle Kommunikation eine Software-Lösung. \\
    \hline
    UART\_RX        & UART\_RX=D0   & Gibt den Port und die Bitnummer für die seriellen  \\
                   &                & Empfangsdaten an. Das Beispiel nimmt               \\
                   &                & Bit 0 des D Ports für den seriellen Eingang.\\
    \hline
    UART\_TX        & UART\_TX=D1   & Gibt den Port und die Bitnummer für die seriellen  \\
                   &                & Sendedaten an.  Das Beispiel nimmt                 \\
                   &                & Bit 1 des D Ports für den seriellen Ausgang.\\
    \hline
    UART           & UART=1         & Wählt für die  serielle Schnittstelle des Chips. \\
                   &                & Eine Auswahl setzt das Verhandensein mehrerer Schnittstellen\\
                   &                & voraus. Nur wirksam ohne die SOFT\_UART Option. \\
    \hline
 LED\_START\_      & LED\_START\_   & Wählt für die Anzahl der Blink-Zyklen \\
   FLASHES         &   FLASHES=3    & für die Kontroll-LED.                            \\
    \hline
 LED               & LED=B3         & Wählt das Port-Bit für die  Kontroll-LED. \\
                   &                & Beim Beispiel würde eine an das Bit 3 des Port B \\ 
                   &                & angeschlossene LED blinken. Bei der \\
                   &                & LED\_START\_FLASHES  Option  blinkt die LED\\
                   &                & die angegebene Anzahl vor dem Kommunikations-Start. \\
                   &                & Mit der LED\_DATA\_FLASH Option leuchtet die LED \\
                   &                & auch während des Wartens auf Daten. \\
    \hline
 LED\_DATA\_       & LED\_DATA\_    & Die Kontroll-LED leuchtet während des Wartens auf \\
      FLASH        &    FLASH=1     & Empfangsdaten der seriellen Kommunikation.\\
    \hline
    \end{tabular}
  \end{center}
  \caption{Wichtige Optionen für die optiboot Makefile}
  \label{tab:options1}
\end{table}

Weitere Optionen sind in der Tabelle~\ref{tab:options2} aufgezählt. 
Diese Optionen sind nur für Software-Untersuchungen und für
Prozessoren ohne Bootloader-Bereich interessant.

\begin{table}[H]
  \begin{center}
    \begin{tabular}{| c | c | l |}
    \hline
   Name der        & Beispiel       & Funktion                                            \\
    Option         &                &                                                     \\
    \hline
    \hline
    SUPPORT\_      & SUPPORT\_      & Wählt für das Bootloader-Programm die Lese- und Schreib- \\
    EEPROM         &  EEPROM=1      & Funktion für EEproms. Wenn als Quelle das Assembler- \\
                   &                & Programm gewählt wurde, ist die EEprom Unterstützung \\
                   &                & ohne gesetzte Option eingeschaltet, kann aber abgeschaltet\\
                   &                & werden, wenn die SUPPORT\_EEPROM Option auf 0 \\
                   &                & gesetzt wird. \\
                   &                & Bei der C-Quelle muß die Funktion mit der Option \\
                   &                & eingeschaltet werden (Standard = aus). \\
    \hline
 C\_SOURCE         & C\_SOURCE=1    & Wählt als Programmquelle das C-Programm anstelle des  \\
                   &                & Assembler-Programms (0 = Assembler).\\
                   &                & Die Assembler Version benötigt weniger Speicherplatz. \\
    \hline
 BIGBOOT           & BIGBOOT=512    & Wählt zusätzlichen Speicherverbrauch für das Bootloader- \\
                   &                & Programm. Das dient nur zum Test der automatischen \\
                   &                & Anpassung an die Programmgröße in der Makefile. \\
    \hline
VIRTUAL\_          & VIRTUAL\_       &  Ändert die Programmdaten eines Anwenderprogramms \\
 BOOT\_            & BOOT\_          & so ab, daß der Bootloader beim Reset angesprochen \\
 PARTITION         & PARTITION       & wird. Für den Start des Anwenderprogramms wird \\
		   &                 & ein anderer Interrupt-Vektor benutzt.          \\
    \hline
 save\_vect\_      & save\_vect\_    & Wählt eine Interrupt-Vektornummer für die  \\
      num          &    num=4        & VIRTUAL\_BOOT\_PARTITION Methode aus.        \\
    \hline
    \end{tabular}
  \end{center}
  \caption{Weitere Optionen für die optiboot Makefile}
  \label{tab:options2}
\end{table}

\section{Benutzung von optiboot ohne Bootloader-Bereich}

Für Prozessoren ohne speziellen Bootloader-Bereich im Flash-Speicher wie dem ATtiny84 ist
eine Möglichkeit vorgesehen, trotzdem optiboot zu benutzen. 
Diese Funktion wird mit der Option VIRTUAL\_BOOT\_PARTITION gewählt.
Dabei wird im Anwenderprogramm auf der Reset-Vektoradresse die Start-Adresse des Bootloaders
eingetragen damit bei einem Reset immer der Bootloader zuerst angesprochen wird.
Die Start-Adresse des Anwender-Programms wird dabei auf die Adresse eines anderen
Interrupt-Vektors verlegt. Dieser Interrupt-Vektor sollte vom Anwenderprogramm nicht benutzt werden.
Wenn der Bootloader in angemessener Zeit keine Daten von der seriellen Schnittstelle
empfängt, springt er zu dem Sprungbefehl, der auf der ,,Ersatz''-Vektoradresse steht und
startet damit das Anwenderprogramm.
Die Abbildung~\ref{fig:VectorMove} soll die Veränderung verdeutlichen.

\begin{figure}[H]
\centering
\includegraphics[]{../FIG/VectorMove.eps}
\caption{Veränderung der Programmdaten durch optiboot}
\label{fig:VectorMove}
\end{figure}

Auf der linken Seite ist der Inhalt der Datei dargestellt, welche die Programmdaten (.hex) enthält.
Rechts daneben ist der Inhalt des Flash-Speichers dargestellt, wie er vom Optiboot Bootloader
geschrieben wird. An zwei Interruptvektor-Adressen wurde der Inhalt verändert.
Einmal wurde auf dem Reset-Vektor 0 der Optiboot Bootloader eingetragen und zum anderen
auf der ,,Ersatz''-Vektor Adresse 4 ein Sprung zum Start des Applikations-Programms eingetragen. 
Eine Schwierigkeit bei der Methode entsteht aber dadurch, daß die Programmdaten nach
der Programmierung meistens zur Kontrolle zurückgelesen werden.
Damit bei der Kontrolle keine Fehler gemeldet werden, gibt Optiboot nicht den wirklichen Inhalt
der Interrupt-Vektortabelle zurück, sondern den Zustand vor seiner Manipulation.
Die Sprung-Adresse im Reset-Vektor kann dafür aus den Daten des ,,Ersatz''-Interruptvektors rekonstruiert werden.
Aber die ursprünglichen Daten des ,,Ersatz''-Interruptvektors wären verloren, da sie an keiner
Stelle der Vektortabelle wiederzufinden sind.
Optiboot benutzt zum Sichern des Original-Inhaltes des ,,Ersatz''-Vektors deshalb die beiden letzten
Bytes des EEprom-Speichers.
Damit ist eine Kontrolle der Programmdaten solange möglich, wie die beiden letzten Bytes
des EEproms nicht überschrieben werden.
Selbst wenn die EEprom Daten überschrieben werden, bleibt der Bootloader funktionsfähig.
Nur die Kontrolle (verify) der Programmdaten ist dann nicht mehr möglich. Bei der Adresse
des ,,Ersatz''-Interruptvektors wird dann ein Fehler gemeldet.

Bei Prozessoren mit mehr als 8 kByte Flash Speicher werden zwei Befehlsworte für jeden Interrupt-Vektor
vorgesehen. Normalerweise stehen auf diesen Doppelworten jmp Befehle mit den jeweiligen Sprungzielen.
Auch diese Art der Vektortabelle wird von Optiboot berücksichtigt. Wenn aber für das Bindeprogramm
(Linker avr-ld) die Option --relax verwendet wird,
werden alle jmp Befehle durch die kürzeren rjmp Befehle ersetzt,
wenn dies bei der jeweiligen Sprungadresse möglich ist.
Dies wird derzeit nicht von optiboot berücksichtigt.
Das Optiboot Programm geht fest davon aus, daß in der Vektortabelle jmp Befehle stehen,
wenn mehr als 8 kByte Flash-Speicher vorhanden sind.
Deshalb wird die VIRTUAL\_BOOT\_PARTITION Methode meistens nicht funktionieren, wenn die --relax Option
beim Programmbinden benutzt wurde!


Weiter ist zu beachten, daß bei Benutzung der VIRTUAL\_BOOT\_PARTITION Option für Prozessoren, die auch
die normale Bootloader Unterstützung bieten, das BOOTRST Fuse Bit nicht aktiviert wird.
Der Grund hierfür ist, daß bei Benutzung der VIRTUAL\_BOOT\_PARTITION die Start-Adresse des Bootloaders
auf einer anderen Adresse liegen kann wie bei der normalen Bootloader Unterstützung.
Bei Benutzung der Option VIRTUAL\_BOOT\_PARTITION kann die Startadresse auf jedem Anfang einer
Seite des Flash-Speichers liegen. Bei der normalen Bootloader Unterstützung kann immer nur das
einfache, doppelte, vierfache oder achtfache einer Mindest-Bootloadergröße berücksichtigt
werden (BOOT\_SZ Fuse-Bits), wie es in Abbildung~\ref{fig:pages} auf Seite~\pageref{fig:pages}
dargestellt wird.
