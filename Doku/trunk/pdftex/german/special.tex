
%\newpage
\chapter{Spezielle Software Teile}

Es wurden zahlreiche Ver\"anderungen durchgef\"uhrt um flash Speicherplatz zu sparen.
Die LCD Ausgabe von Test-Anschlu"snummern wurde in der Form ,,lcd\_data('1'+Pin)'' durchgef\"uhrt.
Um den Platz f\"ur die zus\"atzliche Addition f\"ur jeden Aufruf zu sparen, wurde die
Funktion ,,lcd\_testpin(uint8\_t pin)'' in die Datei lcd\_routines.c eingef\"ugt.


Die Pseudo Funktionen in der Form \_delay\_ms(200) sind keine Bibliotheksfunktionen
sondern es werden f\"ur jede Aufrufstelle Warteschleifen in das Programm eingebaut.
Das verbraucht viel Speicher, wenn man viele Aufrufe an unterschiedlichen Programmstellen hat.
Alle solche Aufrufe wurden durch Aufrufe einer speziellen in Assembler-Sprache geschriebenen
Bibliothek ersetzt, welche nur 74~Bytes des flash-Speichers (bei 8MHz) verbraucht, aber
Aufrufe von wait1us() bis wait5s() in den Stufen 1,2,3,4,5,10,20\dots zur Verf\"ugung stellt.
Die Routinen enthalten den Watch Dog Reset Befehl f\"ur alle Aufrufe \"uber 50ms.
Jeder Aufruf ben\"otigt nur eine Anweisung (2~Byte). Warte-Aufrufe mit Zwischenwerten
wie 8ms ben\"otigen zwei Aufrufe (wait5ms() und wait3ms() oder zwei mal wait4ms() Aufrufe).
Ich kenne keine L\"osung, die \"okonomischer w\"are, wenn man viele Warteaufrufe im Programm benutzt.
Die Aufrufe benutzen keine Register, nur der Stapelzeiger (Stack Pointer) wird f\"ur die R\"uckkehr-Adressen
im RAM (maximal 28 Byte Stack-Platz) benutzt.

Die vollst\"andige Liste der Funktionen ist:\\
wait1us(), wait2us(), wait3us(), wait4us(), wait5us(), wait10us(), \\
wait20us(), wait30us(), wait30us(), wait40us(), wait50us(), wait100us(), \\
wait200us(), wait300us(), wait400us(), wait500us(), wait1ms(),\\
wait2ms(), wait3ms(), wait4ms(), wait5ms(), wait10ms(),\\
wait20ms(), wait30ms(), wait40ms(), wait50ms(), wait100ms(),\\
wait200ms(),wait300ms(), wait400ms, wait500ms(), wait1s(),\\
wait2s(), wait3s(), wait4s() und wait5s();\\
Das sind 36~Funktionen, die nur 37~Anweisungen inklusive dem Watch Dog Reset benutzen!
Da gibt es keinen Weg diese Bibliothek zu verk\"urzen.
Zu guter Letzt halten die Aufrufe die exakte Zeit ein, wenn der unterste Aufruf (wait1us) dies tut.
Nur die Warte-Aufrufe \"uber 50ms sind einen Takt pro 100ms zu lang wegen des zus\"atzlich eingebauten
Watch Dog Reset.


Zus\"atzlich wurde die oft benutzte Folge ,,wait5ms(); ReadADC(\dots);'' durch einen einzelnen Aufruf
,,W5msReadADC(\dots);'' ersetzt.
Dasselbe wurde auch f\"ur die Folge ,,wait20ms(); ReadADC(\dots);'' gemacht, die durch den Aufruf
,,W20msReadADC(\dots);'' ersetzt wurde.
Die Funktion ReadADC wurde zus\"atzlich in Assembler Sprache portiert, so da"s diese Erweiterung
sehr effektiv integriert werden konnte.
Eine funktionell identische C-Version der ReadADC Funktion ist auch beigef\"ugt.
