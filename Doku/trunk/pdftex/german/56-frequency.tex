
%\newpage
\section{Frequenzmessung}
\label{sec:frequency}

Beginnend mit Version 1.10k kann die Frequenzmessung mit einem Bedienmenü angewählt werden.
Die normale Frequenzmessung wird durch Zählen der fallenden Flanken des Eingangssignals T0 (PD4)
mit dem Zähler 0 (COUNTER0) für eine Sekunde erledigt. Für die Einhaltung einer exakten Sekunde
wird der Zähler 1 mit einem 256:1 Vorteiler der CPU-Frequenz benutzt. Der 16-Bit Zähler des ATmega
kann mit dem Vorteiler auch bei 16 Mhz CPU-Frequenz in einem Durchlauf eine Sekunde auszählen.
Für das Starten und das Stoppen des Zählers 0 werden die Vergleichregister B und A des Zählers 1
benutzt. Damit keine zusätzliche Zeitunsicherheit bei den Abfragen entsteht, werden die
Interrupt Service Routinen für beide Vergleichs-Ereignisse benutzt.
Die Zeitverzögerung durch die beiden Interrupt Service Routinen ist ungefähr gleich.
Für die Einhaltung der exakten Sekunde ist eine konstante Zeitverzögerung unerheblich.
Durch Analyse des Assembler Codes kann die Zeitungleichheit ausgeglichen werden.\\

Bei Frequenzen unter \(10 kHz\) wird die Messung durch die Messung einer Periodendauer
ergänzt. Diese Messung wird im Anschluß an eine normale Frequenzmessung durchgeführt.
Dabei werden die Zeit einer bestimmten Anzahl von Pegelwechsel des PCINT20 (PD4) Eingangs
mit dem Zähler 0 gemessen. 
Der Zähler 0 läuft dabei mit voller CPU-Taktrate und das ergibt für eine Periode eine
Aulösung von \(125 ns\). Durch eine größere Anzahl von gemessenen Perioden kann die Auflösung
verbessert werden. Bei 125 Perioden (250 Pegelwechsel) ergeben sich so schon eine mittlere
Auflösung für eine Periode von \(1 ns\). Damit keine Ungenauigkeit beim Starten und
Stoppen von Zähler 0 entsteht, wird der Zähler 0 mit dem ersten Pegelwechsel von
PCINT20 gestartet und nach der vorgegebenen Zahl über die gleiche Interrupt Service Routine
wieder gestoppt.
Die Zahl der Perioden wird so gewählt, daß die Meßzeit etwa 10 Millionen Takte beträgt.
Der Fehleranteil eines Taktes macht dann 0.1ppm aus.
Bei \(8MHz\) beträgt die Meßzeit also etwa 1.25 Sekunden.
Aus der so ermittelten mittleren Periode wird dann eine Frequenz mit höherer Auflösung berechnet.

Zur Kontrolle wurden zwei Tester gegeneinander gemessen.
Erst wurden mit Tester 2 die Testfrequenzen erzeugt und mit Tester 1 die Frequenzen gemessen.
Danach wurde die Messung mit vertauschten Testern wiederholt.
Abbildung \ref{fig:freq-ppm} zeigt die Ergebnisse.
Die nahezu konstanten relativen Abweichungen sind durch die geringen Frequenzunterschiede der beiden Quarze zu erklären.
Eine Abstimmung der Quarzfrequenz wäre mit einstellbaren Kondensatoren am Quarz möglich.
Als Referenz zum Abstimmen könnte beispielsweise das Sekundensignal eines GPS Empfängers dienen.

\begin{figure}[H]
\centering
% GNUPLOT: LaTeX picture with Postscript
\begingroup
  \makeatletter
  \providecommand\color[2][]{%
    \GenericError{(gnuplot) \space\space\space\@spaces}{%
      Package color not loaded in conjunction with
      terminal option `colourtext'%
    }{See the gnuplot documentation for explanation.%
    }{Either use 'blacktext' in gnuplot or load the package
      color.sty in LaTeX.}%
    \renewcommand\color[2][]{}%
  }%
  \providecommand\includegraphics[2][]{%
    \GenericError{(gnuplot) \space\space\space\@spaces}{%
      Package graphicx or graphics not loaded%
    }{See the gnuplot documentation for explanation.%
    }{The gnuplot epslatex terminal needs graphicx.sty or graphics.sty.}%
    \renewcommand\includegraphics[2][]{}%
  }%
  \providecommand\rotatebox[2]{#2}%
  \@ifundefined{ifGPcolor}{%
    \newif\ifGPcolor
    \GPcolortrue
  }{}%
  \@ifundefined{ifGPblacktext}{%
    \newif\ifGPblacktext
    \GPblacktexttrue
  }{}%
  % define a \g@addto@macro without @ in the name:
  \let\gplgaddtomacro\g@addto@macro
  % define empty templates for all commands taking text:
  \gdef\gplbacktext{}%
  \gdef\gplfronttext{}%
  \makeatother
  \ifGPblacktext
    % no textcolor at all
    \def\colorrgb#1{}%
    \def\colorgray#1{}%
  \else
    % gray or color?
    \ifGPcolor
      \def\colorrgb#1{\color[rgb]{#1}}%
      \def\colorgray#1{\color[gray]{#1}}%
      \expandafter\def\csname LTw\endcsname{\color{white}}%
      \expandafter\def\csname LTb\endcsname{\color{black}}%
      \expandafter\def\csname LTa\endcsname{\color{black}}%
      \expandafter\def\csname LT0\endcsname{\color[rgb]{1,0,0}}%
      \expandafter\def\csname LT1\endcsname{\color[rgb]{0,1,0}}%
      \expandafter\def\csname LT2\endcsname{\color[rgb]{0,0,1}}%
      \expandafter\def\csname LT3\endcsname{\color[rgb]{1,0,1}}%
      \expandafter\def\csname LT4\endcsname{\color[rgb]{0,1,1}}%
      \expandafter\def\csname LT5\endcsname{\color[rgb]{1,1,0}}%
      \expandafter\def\csname LT6\endcsname{\color[rgb]{0,0,0}}%
      \expandafter\def\csname LT7\endcsname{\color[rgb]{1,0.3,0}}%
      \expandafter\def\csname LT8\endcsname{\color[rgb]{0.5,0.5,0.5}}%
    \else
      % gray
      \def\colorrgb#1{\color{black}}%
      \def\colorgray#1{\color[gray]{#1}}%
      \expandafter\def\csname LTw\endcsname{\color{white}}%
      \expandafter\def\csname LTb\endcsname{\color{black}}%
      \expandafter\def\csname LTa\endcsname{\color{black}}%
      \expandafter\def\csname LT0\endcsname{\color{black}}%
      \expandafter\def\csname LT1\endcsname{\color{black}}%
      \expandafter\def\csname LT2\endcsname{\color{black}}%
      \expandafter\def\csname LT3\endcsname{\color{black}}%
      \expandafter\def\csname LT4\endcsname{\color{black}}%
      \expandafter\def\csname LT5\endcsname{\color{black}}%
      \expandafter\def\csname LT6\endcsname{\color{black}}%
      \expandafter\def\csname LT7\endcsname{\color{black}}%
      \expandafter\def\csname LT8\endcsname{\color{black}}%
    \fi
  \fi
  \setlength{\unitlength}{0.0500bp}%
  \begin{picture}(7200.00,5040.00)%
    \gplgaddtomacro\gplbacktext{%
      \csname LTb\endcsname%
      \put(814,704){\makebox(0,0)[r]{\strut{}-25}}%
      \csname LTb\endcsname%
      \put(814,1156){\makebox(0,0)[r]{\strut{}-20}}%
      \csname LTb\endcsname%
      \put(814,1609){\makebox(0,0)[r]{\strut{}-15}}%
      \csname LTb\endcsname%
      \put(814,2061){\makebox(0,0)[r]{\strut{}-10}}%
      \csname LTb\endcsname%
      \put(814,2513){\makebox(0,0)[r]{\strut{}-5}}%
      \csname LTb\endcsname%
      \put(814,2966){\makebox(0,0)[r]{\strut{} 0}}%
      \csname LTb\endcsname%
      \put(814,3418){\makebox(0,0)[r]{\strut{} 5}}%
      \csname LTb\endcsname%
      \put(814,3870){\makebox(0,0)[r]{\strut{} 10}}%
      \csname LTb\endcsname%
      \put(814,4323){\makebox(0,0)[r]{\strut{} 15}}%
      \csname LTb\endcsname%
      \put(814,4775){\makebox(0,0)[r]{\strut{} 20}}%
      \csname LTb\endcsname%
      \put(946,484){\makebox(0,0){\strut{} 1}}%
      \csname LTb\endcsname%
      \put(1783,484){\makebox(0,0){\strut{} 10}}%
      \csname LTb\endcsname%
      \put(2619,484){\makebox(0,0){\strut{} 100}}%
      \csname LTb\endcsname%
      \put(3456,484){\makebox(0,0){\strut{} 1000}}%
      \csname LTb\endcsname%
      \put(4293,484){\makebox(0,0){\strut{} 10000}}%
      \csname LTb\endcsname%
      \put(5130,484){\makebox(0,0){\strut{} 100000}}%
      \csname LTb\endcsname%
      \put(5966,484){\makebox(0,0){\strut{} 1e+06}}%
      \csname LTb\endcsname%
      \put(6803,484){\makebox(0,0){\strut{} 1e+07}}%
      \put(176,2739){\rotatebox{-270}{\makebox(0,0){\strut{}Error / ppm}}}%
      \put(3874,154){\makebox(0,0){\strut{}frequency / Hz}}%
    }%
    \gplgaddtomacro\gplfronttext{%
    }%
    \gplbacktext
    \put(0,0){\includegraphics{../GNU/frequency-ppm}}%
    \gplfronttext
  \end{picture}%
\endgroup

\caption{Relativer Fehler für Frequenzmessung }
\label{fig:freq-ppm}
\end{figure}

