\chapter{Beschreibung des Me"sverfahrens}
Ein vereinfachtes Schaltbild eines Eingang/Ausgangs-Pin des ATmega wird in Abbildung~\ref{fig:port} gezeigt.
Der Schalter PUD schaltet die Versorgung f\"ur alle ,,Pull Up'' Widerst\"ande des ATmega ab.
Mit dem Schalter DD kann der Ausgang abgeschaltet werden, der Eingang funktioniert sowohl im Ausgabe wie im
Eingabe-Modus. Im Eingabe-Modus wird mit dem Ausgabe Wert (PORT) der ,,Pull Up'' Widerstand des Eingangs mit geschaltet.
Die beiden Schalter PORT und DD k\"onnen nicht gleichzeitig, sondern nur nacheinander geschaltet werden.
Weil beim Umschalten der ,,Pull Up'' Widerstand die Messung st\"oren k\"onnte, bevorzuge ich die komplette
Abschaltung aller ,,Pull Up'' Widerst\"ande mit dem PUD Schalter.
Nat\"urlich sind die Schalter elektronisch und die Widerst\"ande \(19\Omega\) und \(22\Omega\) sind angen\"aherte Werte.
\begin{figure}[H]
\centering
\includegraphics[]{../FIG/port.eps}
\caption{vereinfachtes Schaltbild jedes ATmega Port Pins}
\label{fig:port}
\end{figure}

Jeder der drei Testpins Ihres TransistorTester wird aus drei ATmega Port Pins gebildet,
was im vereinfachten Schaltbild des Testpins TP2 (mittlerer der drei Pinne) in Abbildung~\ref{fig:terminal} gezeigt wird.

\begin{figure}[H]
\centering
\includegraphics[]{../FIG/terminal.eps}
\caption{vereinfachtes Schaltbild des Testpins TP2}
\label{fig:terminal}
\end{figure}

Jeder Testpin (Me"sport) kann als digitaler oder analoger Eingang benutzt werden.
Diese Me"s\-f\"ahig\-keit ist un\-abh\"an\-gig von der Verwendung des Ports als Ausgang.
Jeder Testpin kann als Ausgang verwendet werden und in diesem Zustand mit GND (0V) oder VCC (5V) verbunden werden,
oder er kann \"uber einen \(680\Omega\) Widerstand oder einen \(470k\Omega\) Widerstand mit entweder GND oder VCC verbunden werden.
Tabelle \ref{tab:case} zeigt alle denkbaren Me"sm\"oglichkeiten.
Beachte, da"s der positive Zustand durch direktes Verbinden mit VCC (Port C) oder
durch Verbinden mit dem \(680\Omega\) Widerstand mit VCC (Port B) erreicht werden kann.
Die gleiche M\"oglichkeit hat der negative Zustand des Testpins zu der GND Seite.
Der Test Zustand meint, da"s der Pin offen sein kann (Eingang), verbunden "uber den \(470k\Omega\) Widerstand
mit VCC oder GND, oder der Pin kann "uber den \(680\Omega\) Widerstand mit VCC oder GND verbunden sein.

\begin{table}[H]
  \begin{center}
    \begin{tabular}{| l | c | c | c |}
    \hline
      & Zustand Pin 1 & Zustand Pin 2 & Zustand Pin 3 \\
    \hline
   1. & positiv    &  negativ   &  test \\
   2. & positiv    &  test      & negativ \\
   3. & test       &  negativ   & positiv \\
   4. & test       &  positiv   & negativ \\
   5. & negativ    &  test      & positiv \\
   6. & negativ    &  positiv   &  test  \\
    \hline
    \end{tabular}
  \end{center}
  \caption{alle Me"sm\"oglichkeiten}
  \label{tab:case} 
\end{table}

Wenn die Kondensatormessung des Testers konfiguriert ist, versucht der Tester vor allen Messungen erst einmal,
die Kondensatoren an allen Anschlu"spins zu entladen. Wenn das nicht gelingt, also die Restspannung zu hoch bleibt,
wird das Entladen nach etwa 12 Sekunden mit der Meldung ,,Cell!'' abgebrochen. Dies kann auch dann vorkommen, wenn
gar kein Kondensator angeschlossen ist. Die Ursache kann in diesem Fall sein, da"s die Entlade-Grenzspannung f\"ur diesen
ATmega zu niedrig gew\"ahlt ist. Man kann eine h\"ohere Restspannung mit der Makefile Option CAP\_EMPTY\_LEVEL w\"ahlen.
