\chapter{Bedienungshinweise}
\label{sec:manual}
Die Bedienung des Transistortesters ist einfach.
Trotzdem sind einige Hinweise erforderlich.
Meistens sind an die drei Testports über Stecker-Leitungen mit Krokodilklemmen oder anderen Klemmen angeschlossen.
Es können auch Fassungen für Transistoren angeschlossen sein.
In jedem Fall können Sie Bauteile mit drei Anschlüssen mit den drei Testports in beliebiger Reihenfolge verbinden.
Bei zweipoligen Bauteilen können Sie die beiden Anschlüsse mit beliebigen Testports verbinden.
Normalerweise spielt die Polarität keine Rolle, auch Elektrolytkondensatoren können beliebig angeschlossen werden.
Die Messung der Kapazität wird aber so durchgeführt, dass der Minuspol am Testport mit der kleineren Nummer liegt.
Da die Messspannung aber zwischen 0,3 V und maximal 1,3 V liegt, spielt auch hier die Polarität keine wichtige Rolle.
Wenn das Bauteil angeschlossen ist, sollte es während der Messung nicht berührt werden. Legen Sie es auf einen
isolierenden Untergrund ab, wenn es nicht in einem Sockel steckt. Berühren Sie auch nicht die Isolation der Messkabel,
das Messergebnis kann beeinflusst werden.
Dann sollte der Starttaster gedrückt werden.
Nach einer Startmeldung erscheint nach circa zwei Sekunden das Messergebnis. Bei einer Kondensatormessung kann es
abhängig von der Kapazität auch deutlich länger dauern.

Was dann weiter geschieht, hängt von der Softwarekonfiguration des Testers ab.
\begin{description}
  \item[Einzelmessung] Wenn der Tester für Einzelmessung konfiguriert ist, schaltet der Tester nach einer Anzeigezeit von 10 Sekunden
(konfigurierbar) wieder automatisch aus, um die Batterie zu schonen. Während der Anzeigezeit kann aber
auch vorzeitig eine neue Messung gestartet werden. Nach der Abschaltung kann natürlich auch wieder eine
neue Messung gestartet werden, entweder mit dem gleichen Bauteil oder mit einem anderen Bauteil.\\

  \item[Dauermessung] Einen Sonderfall stellt die Konfiguration ohne die automatische Abschaltfunktion dar.
Diese Konfiguration wird normalerweise nur ohne die Transistoren für die Abschaltung benutzt.
Es wird stattdessen ein externer Ein-/Aus-Schalter benötigt. Hierbei wiederholt der Tester die
Messungen solange, bis ausgeschaltet wird.\\

  \item[Serienmessung] In diesem Konfigurationsfall wird der Tester nicht nach einer Messung sondern erst nach einer konfigurierbaren
Zahl von Messungen abgeschaltet. Im Standardfall wird der Tester nach fünf Messungen ohne erkanntes Bauteil
abgeschaltet. Wird ein angeschlossenes Bauteil erkannt, wird erst bei der doppelten Anzahl, also zehn Messungen abgeschaltet.
Eine einzige Messung mit nicht erkanntem Bauteil setzt die Zählung für erkannte Bauteile auf Null zurück.
Ebenso setzt eine einzige Messung mit erkannten Bauteil die Zählung für die nicht erkannten Bauteile auf Null zurück.
Dies hat zur Folge, dass auch ohne Betätigung des Starttasters immer weiter gemessen werden kann,
 wenn Bauteile regelmässig gewechselt werden.
Ein Bauteilwechsel führt in der Regel durch die zwischenzeitlich leeren Klemmen zu einer Messung ohne erkanntes Bauteil.\\

Eine Besonderheit gibt es in diesem Betriebsmodus für die Anzeigezeit. Wenn beim Einschalten der Starttaster nur kurz
gedrückt wurde, beträgt die Anzeigezeit der Messergebnisse nur drei Sekunden. Wenn der Starttaster bis zum Erscheinen der
ersten Meldung festgehalten wurde, beträgt die Anzeigezeit wie bei der Einzelmessung zehn Sekunden.
Ein vorzeitiger neuer Messbeginn ist aber während der Anzeigezeit durch erneutes Drücken des Starttasters möglich.\\

\end{description}

Wenn die Software mit der Selbsttestfunktion konfiguriert ist, kann der Selbsttest durch einen Kurzschluss aller drei
Testports und drücken der Starttaste gestartet werden.
Hier werden die im Selbsttest-Kapitel \ref{sec:selftest} beschriebenen Tests ausgeführt. Die viermalige Testwiederholung
kann vermieden werden, wenn der Starttaster gedrückt gehalten wird. So kann man uninteressante Tests schnell beenden und
sich durch Loslassen des Starttasters interessante Tests viermal wiederholen lassen.
Wenn die Funktion AUTO\_CAL in der Makefile gewählt ist, wird beim Selbsttest
eine Kalibration des Innenwiderstandes der Port-Ausgänge und
eine Kalibration der Nullwertes für die Kondensatormessung durchgeführt.
Sie sollten während der Kalibration werder die Testports noch angeschlossene Kabel berühren. Die Ausrüstung
sollte aber die gleiche sein, die später zum Messen verwendet wird.
Anderenfalls wird der Nullwert der Kondensatormessung nicht richtig bestimmt.
Wenn Sie während Test Nummer 10 (nach der Ausgabe der Nullwerte der Kondensatormessung) einen Kondenstor mit einer beliebiger Kapazität zwischen \(100 nF\) und \(20 \mu F\) an Pin~1 und Pin~3  anschliessen,
wird die Offset Spannung des analogen Komparators kompensiert, um genauere Kapazit"atswerte ermitteln zu können.
Die Verstärkung für ADC Messungen mit der internen Referenz-Spannung wird dann ebenfalls abgeglichen, um
bessere Widerstands-Messergebnisse mit der AUTOSCALE\_ADC Option zu erreichen.


Der Nullwert für die ESR-Messung wird als Option ESR\_ZERO in der Makefile vorbesetzt. Dieser Nullwert, der im
Normalfall zu hoch gesetzt ist, wird von der Software im EEprom vorbesetzt und wird bei jedem Selbsttest
wieder auf diesen Wert zurückgesetzt.
Nach jeder ESR-Messung wird geprüft, ob das Resultat negativ ist. In diesem Fall (Ausgabe von ,,ESR=0?'') wird der Nullwert so weit
reduziert, dass sich bei der nächsten Messung ein ESR-Wert von Null ergibt. Damit kann ein Nullabgleich mit
einem Elektrolytkondensator hoher Kapazität und mit niedrigem ESR durchgeführt werden.
Dieser ESR-Nullabgleich muss aber nach jedem Selbsttest einige Male wiederholt werden.
