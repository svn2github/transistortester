\chapter{Configuring the TransistorTester}
\label{sec:config}
The complete software for the TransistorTester is available in source code.
The compilation of modules is controlled with a Makefile. The developement was done
at the Ubuntu Linux operating system with the GNU toolchain (gcc version 4.5.3).
It should be possible to use other Linux operating systems without problems.
To load the compiled data to the flash memory or
the EEprom memory, the tool avrdude (version 5.11svn) was taken by the Makefile, if you call ``make upload''.
 The program avrdude \cite{avrdude} is available for Linux and Windows operating system.
The gnu C-compiler gcc is also taken by the AVR studio software and
by the WinAVR \cite{winavr1},\cite{winavr2} software at the Windows operating system.
You can load the program data (.hex and .eep) also with other tools to the ATmega,
but only my Makefile version takes care to load the correct data to the choosed processor.
Avrdude loads only data to the ATmega if the Signature Bytes of the connected ATmega is
identical to the choosed one. 
If you alter the Makefile, all the software will be compiled new, if you call a ``make'' or
``make upload'' command. The software compiled for a ATmega8 does not run on a ATmega168.
The software compiled for a ATmega328 does not run on the ATmega168! 
A exeption from this rule is the software compiled for ATmega168, this data can also be used
for a ATmega328 without changes.
Be careful, if you don't use my Makefile.

With the correct options set, my software runs on the unchanged hardware of Markus F.
You must set the option PARTNO=M8, {\bf NOT} the option NO\_AREF\_CAP and {\bf NOT} the  PULLUP\_DISABLE option.
The clock rate can also be set to \(8MHz\) with fuses, no crystal is required!


The following options in the Makefile are avaiable to configure the software for your Tester.

\begin{description}
  \item[PARTNO] describes the target processor:\\
         m8 = ATmega8\\
         m168 or m168p = ATmega168\\
         m328 or m328p = ATmega328\\
         m644 or m644p = ATmega644\\
         m1284p        = ATmega1284\\
         m1280         = ATmega1280\\
         m2560         = ATmega2560\\
    Example:  PARTNO = m168
   \item[UI\_LANGUAGE] specifies the favored Language\\
    LANG\_BRASIL, LANG\_CZECH, LANG\_DUTCH, LANG\_ENGLISH, LANG\_GERMAN, \\
    LANG\_HUNGARIAN, LANG\_ITALIAN, LANG\_LITHUANIAN, LANG\_POLISH, \\
    LANG\_RUSSIAN, LANG\_SLOVAK, LANG\_SLOVENE, LANG\_SPANISH  and \\
    LANG\_UKRAINIAN is currently avaiable.
 The russian or ukrainian language requires a LCD with cyrillic character set.\\
    Example:  UI\_LANGUAGE = LANG\_ENGLISH

  \item[LCD\_CYRILLIC] is only needed for a LCD-display with cyrillic character set. The \(\mu\) and \(\Omega\) character
is not avaiable with the cyrillic character set.
If you specify this option, both characters are loaded to the LCD with software.
You should set this option, if your display shows wrong characters instead of \(\mu\) or \(\Omega\).\\
Example: CFLAGS += -DLCD\_CYRILLIC

 \item[LCD\_DOGM] must be set, if a LCD with ST7036 controller (Type DOG-M) is used for displaying.
The LCD-contrast is then set with software commands.
If you have changed the contrast value to a wrong value, so that you can not read anything at your display,
you shouls first try to read somthing from a side look to the display.
If this fails also, you should reset the EEprom to the initial values with a ISP programmer.\\
Example: CFLAGS += -DLCD\_DOGM

 \item[FOUR\_LINE\_LCD] can be used with a 4x20 character display for better using the additional space.
Additional parameters, which are shown only short in row 2, will be shown in row 3 and 4 with this option.\\
Example: CFLAGS += -DFOUR\_LINE\_LCD

  \item[DD\_RAM\_OFFSET] Somne character displays use different DD-RAN starting addresses for the beginning of each line.
Usually the DD-RAM starting address for line 1 is 0.
Some displays like TC1604 or TC1602 use a 128 (0x80) for the beginning of line 1.
This can be respected with this option.
Example: CFLAGS += -DDD\_RAM\_OFFSET = 128

  \item[WITH\_LCD\_ST7565] This option must be used, if a 128x64 pixel LCD is connected with serial
interface. For this display type further options must be set, which are described in table~\ref{tab:cod-display}.
You can also use the simular SSD1306 controller instead of the ST7565 controller for example.
This must be done by setting the variable WITH\_LCD\_ST7565 to 1306.
A PCF8812 or PCF8814 Controller is also supported, if the Option is set correctly.
Also a display with a ST7920 or NT7108 controller can be connected.
For the NT7108 controller a additional serial-parallel converter 74HC(T)164 or 74HC(T)595 must be used.  \\
Example: WITH\_LCD\_ST7565 = 1 

 \item[LCD\_INTERFACE\_MODE] For the SSD1306 controller also the I\textsuperscript{2}C type interface with address 0x3c
can be used  instead of the 4-wire SPI interface by setting this option to 2.
For the ST7920 controller a special serial interface can be selected by setting this option to 5.
If only one connection type is provided for a controller, you need not set the constant LCD\_INTERFACE\_MODE .
All currently used values for LCD\_INTERFACE\_MODE and WITH\_LCD\_ST7565 are shown in table~\ref{tab:cod-display}. \\

\begin{table}[H]
  \begin{center}
    \begin{tabular}{| c | c | c | c|}
    \hline
 Display-Type       &  Interface       & WITH\_LCD\_ST7565 &  LCD\_INTERFACE\_MODE \\
    \hline
    \hline
  Character 16x2,   & 4-Bit parallel   &  disabled (0)     & disabled (1) \\
  Character 20x4    &  4-Bit SPI       &                   &    4   \\
                  & I\textsuperscript{2}C &                &   2    \\
    \hline
  Graphic ST7565    & 4-Bit SPI        &   1 or 7565       &  disabled (4) \\
    \hline
  Graphic ST7565  & I\textsuperscript{2}C & 1 or 7565      &   2 \\
    \hline
  Graphic SSD1306   & 4-Bit SPI        &   1306            &  disabled (4) \\
    \hline
  Graphic SSD1306  & I\textsuperscript{2}C & 1306          &   2 \\
    \hline
  Graphic ST7920    & 4-Bit parallel   &   7920            &  disabled (1) \\
    \hline
  Graphic ST7920    & 2-Bit serial     &   7920            &  5 \\
    \hline
  Graphic NT7108    & 8-Bit parallel   &   7108            &  disabled (6) \\
    or KS0108       &    + 74HCT164    &                   &      \\
    \hline
  Graphic PCF8812   & SPI              &   8812            & disabled (4) \\
    \hline
  Graphic PCF8814   & SPI              &   8814            & disabled (4) \\
                  & I\textsuperscript{2}C & 8814           &   2 \\
                    & 3-line           &   8814            &   3 \\
    \hline
  Graphic ILI9163   & SPI             & 9163              & disabled (4) \\
  128x128 Color     &                 &                   &              \\
    \hline
  Graphic ST7735    & SPI             & 7735              & disabled (4) \\
  128x160 Color     &                 &                   &              \\
    \hline
    \end{tabular}
  \end{center}
  \caption{Number setting for controller and interface mode}
  \label{tab:cod-display}
\end{table}

The values in brackets are used software internal and are shown for information only.
You should not set the values in brackets here in the Makefile.\\

Example: CFLAGS += -DLCD\_INTERFACE\_MODE=2

  \item[LCD\_SPI\_OPEN\_COL] With the option LCD\_SPI\_OPEN\_COL the data signals of the SPI interface
are not switched to VCC directly.
The signals are switched to GND only, for high signals the pullup resistors of the ATmega are used.
For the RESET signal a external pull-up resistor is required, if the option PULLUP\_DISABLE is set.
For the other signals the internal pullup resistors of the ATmega are temporary used,
even if option PULLUP\_DISABLE is set.
Example: CFLAG += -DLCD\_SPI\_OPEN\_COL

\item[LCD\_I2C\_ADDR] The I\textsuperscript{2}C address of the SSD1306 controller can be selected to 0x3d by presetting
the constant LCD\_I2C\_ADDR to 0x3d.\\
Example: CFLAGS += -DLCD\_I2C\_ADDR=0x3d

  \item[LCD\_ST7565\_RESISTOR\_RATIO] With this option the resistor ratio for the voltage regulator of
the ST7565 controller is set.
Usually values between 4 and 7 are practical.
The value can be set between 0 and 7.\\
Example: LCD\_ST7564\_RESISTOR\_RATIO = 4

  \item[LCD\_ST7565\_H\_FLIP] With this option the display content can be flipped in horizontal direction.\\
Example: CFLAGS += -DLCD\_ST7565\_H\_FLIP = 1

  \item[LCD\_ST7565\_H\_OFFSET] This option can be used to adapt the display window to the used memory area.
 The controller uses more horizontal pixel (132) as the display window shows (128).
 Depending of your display module a value of 0, 2 or 4 can be required for proper presentation.\\
Example: CFLAGS += -DLCD\_ST7565\_H\_OFFSET = 4

  \item[LCD\_ST7565\_V\_FLIP] With this option the display content can be flipped in vertical direction.\\
Example: CFLAGS += -DLCD\_ST7565\_V\_FLIP = 1

  \item[VOLUME\_VALUE] You can predefine a contrast value for ST7565 or SSD1306 controllers.
The value for the ST7565 controller can be between 0 and 63. For the SSD1306 controller you can
select a value between 0 and 255.\\
Example: CFLAGS += -DVOLUME\_VALUE = 25

  \item[LCD\_ST7565\_Y\_START] With this option you can set the first row correctly to the top of screen.
The first row is shifted to the middle of the screen for some display variants.
For this variants you can shift the first row to the top of the screen again,
if this option is set to 32 (half of the screen height).\\
Example: CFLAGS += -DLCD\_ST7565\_Y\_START = 32

  \item[LCD\_CHANGE\_COLOR] This option expand the menu functions with a selection item
to change the background and the foreground color.
If the value is set to 2, the colors blue and red are swapped.
You can select this option only for color displays (controller ST7735 or ILI9163).\\
Example: CFLAGS += -DLCD\_CHANGE\_COLOR=1

 \item[LCD\_BG\_COLOR] With this 16-bit value you can select a background color.
Normally the upper 5 bits are used for the color red, the middle 6 bits are used for the color green
and the lower 5 bits are used for the color blue. Sometimes the bits for the colors red and blue
are swapped.
You can select this option only for color displays (controller ST7735 or ILI9163).\\
Example: CFLAGS += -DLCD\_BG\_COLOR=0x000f

 \item[LCD\_FG\_COLOR] With this 16-bit value you can select a forground color.
The example selects the color white for text and symbols.
You can select this option only for color displays (controller ST7735 or ILI9163).\\
Example: CFLAGS += -DLCD\_FG\_COLOR=0xffff

  \item[FONT\_8X16] You must select one font size for the ST7565 controller.
Selectable are different fonts with the name ''FONT\_'' with appended size information (width X height).
Currently the font sizes 6X8, 8X8, 7X12, 8X12, 8x12thin, 8X14, 8X15, 8X16, and 8X16thin are available.
Font size 8X16 or 8x16thin is the most efficient use of graphics space for a 128x64 pixel LCD.\\
Example: FONT\_8X16

 \item[BIG\_TP] The pin numbers for the graphical presentation can be shown bigger with this option.\\
Example: CFLAGS += BIG\_TP

 \item[INVERSE\_TP] With this option you can select a inverse presentation (white background) of the pin numbers
on the graphical display.
Because a boarder in required for this presentation, you can not combine this option with the BIG\_TP option.\\
Example: CFLAGS += INVERSE\_TP

  \item[STRIP\_GRID\_BOARD] This option adapts the software to a changed port D connection for strip grid printed boards.
You can find the details in the chapter hardware \ref{sec:hardware} at page~\pageref{sec:hardware}.
You can also choose alternative assignments of ATmega pins for graphical displays.
For the chinese ''T5'' board you must set the STRIP\_GRID\_BOARD option to 5.
For alternative pin assignments of graphical displays the assignment of the pushbutton signal is unchanged.\\
Example: CFLAGS += -DSTRIP\_GRID\_BOARD

  \item[WITH\_MENU] activated a menu function for a ATmega328. You can select some additional functions with a
selection menu, which you can call with a long key press (\textgreater~0.5s).\\
Example: CFLAGS += -DWITH\_MENU

 \item[MAX\_MENU\_LINES]
This option specifies a maximum count of lines for the shown choices of menu items.
Normally the count of lines for the menu items is given by the present count of lines of the display.
Because there are usually more items selectable as the display can support,
the choices are replaced in a cyclic manner.
Building the display content in this cyclic way will take several time, especially for big color displays 
with many lines.
With the limitation of the line count by this option you can reduce the output time for the menu choices significant,
which will speed up the operation.
The default value for this item is 5.\\
Beispiel: CFLAGS += -DMAX\_MENU\_LINES=3


  \item[WITH\_ROTARY\_SWITCH] The menu function can be easier controlled with a the extension of a rotary pulse encoder.
See the description~\ref{fig:RotExt} in the Hardware section for details of the required extension.
If your rotary pulse encoder has the same count of indexed positions (detent) as pulses of the switch for every turn, you must
set the  option WITH\_ROTARY\_SWITCH to 2. If the rotary pulse encoder has twice the count of indexed position, you must
set the option WITH\_ROTARY\_SWITCH to 1.
Setting the WITH\_ROTARY\_SWITCH to 5 selects the highest resolution for the rotary switch. Every cycle of the two switches results
to a count of 4. Usually this setting is only usefull for rotary switch encoders without indexed positions.
A setting of the WITH\_ROTARY\_SWITCH to 4 is required for correct handling of two separate push buttons for Up and Down,
which are installed instead of the normal rotary encoder switches.
Do not use a setting of 4 for normal rotary encoders!\\
Example: CFLAGS += -DWITH\_ROTARY\_SWITCH=1

  \item[CHANGE\_ROTARY\_DIRECTION] You can change the direction of the detected rotary direction by hardware swap of
the two switch signals or by setting the this option.\\
Example: CFLAGS += -DCHANGE\_ROTARY\_DIRECTION

  \item[WITH\_SELFTEST] If you specify this Option, software will include a selftest function.
Selftest will be started, if you connect all three probes together and start measurement.
If the menu function is selected, only the calibration part of the self test is executed by automatic start with
shorted probes. The selftest parts T1 to T7 are only executed, if the selftest is started with menu selection.\\
Example: CFLAGS += -DWITH\_SELFTEST

  \item[NO\_COMMON\_COLLECTOR\_HFE] disables the hFE measurement of transistors with the common collector circuit.
You can save memory to enable the extended selftests T1 to T7 for a ATmega168 processor.
By default both measurement circuits for the hFE measurement are enabled, 
but there is no place in the program memory of the ATmega168 for the extended selftests.\\
Example: CFLAGS += -DNO\_COMMON\_COLLECTOR\_HFE

  \item[NO\_COMMON\_EMITTER\_HFE] disables the hFE measurement of transistors with the common emitter circuit.
You can save memory to enable the extended selftests T1 to T7 for a ATmega168 processor.
By default both measurement circuits for the hFE measurement are enabled, 
but there is no place in the program memory of the ATmega168 for the extended selftests.\\
Example: CFLAGS += -DNO\_COMMON\_EMITTER\_HFE

  \item[NO\_TEST\_T1\_T7] This option disable the execution of the selftest parts T1 to T7.
This tests are usefull to find errors in the hardware like incorrect measurement resistors or isolation problems.
If your hardware is well, you can omitt this selftest parts T1 to T7 by setting this option to get a faster calibration.
With enabled menu function the selftest parts T1 to T7 are only started by selection of the menu function ''Selftest''.
The ATmega168 processor does not use the selftest parts T1 to T7, if both measurement types for hFE determination are used.\\
Example: CFLAGS += -DNO\_TEST\_T1\_T7

  \item[AUTO\_CAL] The zero offset for capacity measurement will be written additionally
to the EEprom with the selftest routine. Additionally the offset voltage of the analog comparator (with option REF\_C\_KORR) and the
voltage offset of the internal reference voltage (REF\_R\_KORR) will be measured automatically, if you connect a
capacitor with a capacity value between \(100nF\) and \(20\mu F\) to pin~1 and pin~3 after measurement of capacity zero offset. 
All found values will be written to EEprom and will be used for further measurements automatically.
The port output resistance values will be determined at the beginning of each measurement.\\
Example: CFLAGS += -DAUTO\_CAL

  \item[SHORT\_UNCAL\_MSG] After the test of a part a message is shown for processors with at least 32K flash memory,
if the tester is still uncalibrated. Normally followes after the hint a short description, how the
tester can be calibrated. This description is not shown, if you set the option SHORT\_UNCAL\_MSG in the Makefile.
With this option set, the tester only display a one line hint.
This reduces the required space of flash memory  and also the display time for the user,
which already know, how to calibrate the tester.\\
Example: CFLAGS += -DSHORT\_UNCAL\_MSG

 \item[NO\_ICONS\_DEMO]
This option will switch off the additional demonstration of the icons and the output of the character set with
the menu function ''Show data''.
This reduces the required space of flash memory  and also the display time for the user.\\
Example: CFLAGS += -DNO\_ICONS\_DEMO

 \item[WITH\_ROTARY\_CHECK]
This option enables the additional menu function for the test of a rotary encoder.
For the test you must connect a rotary encoder to the test pins TP1, TP2 and TP3.
Please note, that you can not check the build-in rotary-encoder of the tester!
You can also use a rotary encoder for ease in operation of the tester with the option WITH\_ROTARY\_SWITCH.\\
Example: CFLAGS += -DWITH\_ROTARY\_CHECK

 \item[NO\_FREQ\_COUNTER]
With this option you can deselect the frequency counter function of the tester.
This is especially useful if the pin PD4 (ATmega328) can not be used together with the
connected display.
The corresponding entry in the list of menu functions then no longer appears and also
Flash memory space is saved.\\
Example: CFLAGS += -DNO\_FREQ\_COUNTER

 \item[WITH\_FREQUENCY\_DIVIDER]
With this option the menu is expanded by a selectable prescaler for the frequency counter.
The scaler can be selected to 1:1, 1:2, 1:4, 1:8, 1:16, 1:32, 1:64 and 1:128 .
This option is only useful, if a external prescaler can be connected to the frequency input of the tester.
The shown frequencies and periodes of the measurements will respect the selected scaling factor.\\
Example: CFLAGS += -DWITH\_FREQUENCY\_DIVIDER

  \item[WITH\_SamplingADC] With this option set, the tester make use of the sampling method of ADC in special cases.
By shifting the sampling time of the ADC with increments of 1, 4 or 16 processor clock intervals for repeatable signals
fast changes of voltages can be monitored.
The load time of little capacitors below \(100pF\) can be monitored with a resulting resolution of \(0.01pF\) with a \(16MHz\) processor clock.
With the same method the resonant frequency of little coils below \(2mH\) can be monitored with a parallel capacitor to build a LC-resonator.
If the capacity of the parallel capacitor is known, the inductance of the coil can be calculated with high resolution from
the resonant frequency. As a side product the quality factor Q can be estimated from the resonant behavior.
This features are switched on by setting the option WITH\_SamplingADC.
At the calibration sequence additionally the zero capacity values of the sampling method is measured and
after that the capacity value of a suitable capacitor for later building the LC-resonator with a unknown coil is measured.\\
Example: WITH\_SamplingADC = 1

  \item[WITH\_XTAL]
This option enables additional tests for crystals and resonators, if the SamplingADC function is also enabled and
a 16~MHz crystal is used for clock generation (OP\_MHZ = 16).
If possible, the frequencies for serial and parallel circuit is measured and than the serial capacity Cm of the
equivalent circuit is tried to compute from the frequency offset.\\
Example: CFLAGS += -DWITH\_XTAL

  \item[WITH\_UJT]
This option enables additional tests for  Unijunction transistors. 
If the SamplingADC function is enabled, the tester tries to build a oscillator with the part.
But the UJT type is also detected without the SamplingADC function.
Without the option WITH\_UJT the unijunction transistors are detectes as double diode.\\
Example: CFLAGS += -DWITH\_UJT

  \item[WITH\_PUT]
This option enables a additional test for ''Programmable Unijunction Transistor''.
Without this option PUTs are usually detected as Bipolar Junction Transistor.\\
Example: CFLAGS += -DWITH\_PUT

 \item[FET\_Idss]
This option enables additional measurements to compute the drain current Idss, if the estimation is not
above \(60mA\). The estimation and calculation is done with a assumed quadratical current propagation.\\
Example: CFLAGS += -DFET\_Idss

  \item[FREQUENCY\_50HZ] At the end of selftest a 50~Hz Signal will be generated on Port~2 and Port~3 for up to one minute.
 This option should be set only for special cases to check the delay function.\\
Example: CFLAGS += -DFREQUENCY\_50HZ

  \item[CAP\_EMPTY\_LEVEL]  This option defines the voltage level for discharged capacitor (mV units).
You can set the level to higher value as \(3mV\), if the tester does not finish discharging of capacitors.
In this case the tester ends after longer time with the message ``Cell!''.\\
Example: CFLAGS += -DCAP\_EMPTY\_LEVEL=3

  \item[WITH\_AUTO\_REF] specifies, that reference voltage is read to get the actual factor for capacity measuring of low capacity values (below \(40\mu F\)).\\
Example:  CFLAGS += -DWITH\_AUTO\_REF

  \item[REF\_C\_KORR] specifies a offset for readed reference voltage in mV units.
This can be used to adjust the capacity measurement of little capacitors.
A correction value of 10 results to about 1~percent lower measurement results.
If the option AUTO\_CAL is selected together with the WITH\_SELFTEST option, the REF\_C\_KORR will be
a offset to the measured voltage difference of the test capacitor and the internal reference voltage.\\
Example:  CFLAGS += -DREF\_C\_KORR=14

  \item[REF\_L\_KORR] specifies a additional offset in mV units to the reference voltage for the measurement of
inductance values. 
The REF\_C\_KORR offset and respectively the offset value from the calibration is additionally used with the inductance measurement.
The REF\_L\_KORR value will be subtracted for measurements without a \(680\Omega\) resistor,
for measurements with a \(680\Omega\) resistor the value will be added.
A correction value of 10 will change the result about 1 percent.\\
Example: CFLAGS += -DREF\_L\_KORR=40

  \item[C\_H\_KORR] specifies a correction value for the measurement of big capacitor values.
A value of 10 results to 1~percent lower measurement results.\\
Example:  CFLAGS += -DC\_H\_KORR=10

  \item[WITH\_UART] uses the pin PC3 as output for the serial text (V24).
If the option is not set, the pin PC3 can be used for reading a external voltage with a 10:1 resistor divider.
With this equipment you can check the breakdown voltage of zener diodes, which have more than \(4.5V\) breakdown voltage.
This measurement will repeat with 3 measurements per second until you release the Start button.\\
Example: CFLAGS += -DWITH\_UART

  \item[TQFP\_ADC6] The Option TQFP\_ADC6 uses the additional input ADC6 of the ATmega with TQFP or QFN package instead of
the PC3 pin (ADC3).
With this option the external voltage input can be used independent of the usage of PC3 pin for serial output.
The ADC6 input is then used for the zener diode measurement and for the dialog selectable external voltage measurement
for a ATmega328.\\
Example: CFLAGS += -DTQFP\_ADC6

  \item[TQFP\_ADC7] The Option TQFP\_ADC7 uses the additional input ADC6 of the ATmega with TQFP or QFN package instead of
the PC3 pin (ADC3).
With this option the external voltage input can be used independent of the usage of PC3 pin for serial output.
If this option is used without the option TQFP\_ADC6, both the zener diode measurement and the measurement of external voltage
with the dialog is done with the ADC7 analog input.
If this option is used together with the TQFP\_ADC6 option, is the zener diode measurement done with the ADC6 pin and
both pins are used for voltage measurement with the dialog of the ATmega328.
Both ADC input pins shouls be assembled with a 10:1 voltage divider.\\
Example: CFLAGS += -DTQFP\_ADC7

  \item[WITH\_VEXT] enables the measurement of a external voltage with a 10:1 voltage divider.
For the ATmega168 or ATmega328 processor usually the PC3 pin is used as input, if no option TQFP\_ADC6 or
TQFP\_ADC7 is set. In this case this option is only possible, if the WITH\_UART option is not set.\\
Example: CFLAGS += -DWITH\_VEXT 

  \item[RMETER\_WITH\_L] select for the resistor measurement function, which is selected with a resistor at TP1 and TP3,
additionally the measurement of inductance. The operation mode is indicated with a {\bf[RL]} at the end of the first display line.
With the additional test for inductance the measurement time is increased for resistors below \(2100\Omega\) considerably.
Also resistors below \(10\Omega\) will not be measured with the ESR methode Without this option, because
a part with inductance can not be excluded. Because the ESR measurement method  uses short current pulses,
parts with inductance can not be measured. The resistors below \(10\Omega\) can only measured with a resolution of
\(0.1\Omega\) without this option, because only with the ESR method a resolution of \(0.01\Omega\) can be obtained.
If this option is set, the previous limitations are not affected, but the measurement time can be longer.\\
Example: CFLAGS += -DRMETER\_WITH\_L

  \item[AUTOSCALE\_ADC] enables the automatic scale switchover of the ADC to either VCC or internal reference.
Internal reference gives a \(2.56V\) scale for ATmega8 and a \(1.1V\) scale for other processors.
For the ATmega8 the automatic scale switchover is not used any more.\\
Example: CFLAGS += -DAUTOSCALE\_ADC

  \item[ESR\_ZERO] defines a zero offset for ESR measurements.
The zero offsets for all three pin combinations will be determined with the selftest and replaces the preset zero offset.
This zero offsets will be subtracted from all ESR measurements.\\
Example: CFLAGS += -DESR\_ZERO=29

  \item[NO\_AREF\_CAP] tells your Software, that you have no capacitor (\(100nF\)) installed at pin AREF (pin 21).
This enables a shorter wait-time for the AUTOSCALE\_ADC scale switching of the ADC.
A \(1nF\) capacitor was tested in this mode without detected errors.
Figure \ref{pic:aref1} and \ref{pic:aref5} show the switching time with a \(1nF\) capacitor.
As you can see the switching from \(5V\) to \(1.1V\) is much slower than switching back to \(5V\). If you
have still installed the \(100nF\), switching time will be about factor 100 longer!\\
Example: CFLAGS += -DNO\_AREF\_CAP

  \item[REF\_R\_KORR] specifies a offset for the internal ADC-reference voltage in mV units.
With this offset a difference by switching from VCC based ADC reference to internal ADC reference for resistor measurement can be adjusted.
If you select the AUTO\_CAL option of the selftest section, this value is only a additionally offset to the found voltage 
difference in the AUTO\_CAL function.\\
Example: CFLAGS += -DREF\_R\_KORR=10

  \item[OP\_MHZ] tells your software at which Clock Frequency in MHz your Tester will operate.
The software is tested only for \(1MHz\), \(8MHz\) and additionally \(16MHz\). 
The \(8MHz\) operation is recommended for better resolution of capacity and inductance measurement.\\
Example: OP\_MHZ = 8

  \item[RESTART\_DELAY\_TICS] must be set to 6, if the ATmega168 or ATmega328 is used with the internal RC-oszillator instead of
the crystal oszillator.
If this value is not preset, the software respects the 16384 clock tics delay for restart from sleep mode with the crystal operation.\\
Example: CFLAGS += -DRESTART\_DELAY\_TICS=6

  \item[USE\_EEPROM] specifies if you wish to locate fix text and tables in EEprom Memory. Otherwise the flash memory is used.
Recommended is to use the EEprom (option set).\\
Example: CFLAGS += -DUSE\_EEPROM

\item[EBC\_STYLE] specifies, that the output of transistor pin layout is done with format ``EBC=...'' or ``GDS=...''.
This way of output save program memory for the ATmega. Without this option the layout is shown with the
format ``123=...'', where every point represent a E~(Emitter), B~(Base) or C~(Collector).
For FET transistors every point can be a G~(Gate), D~(Drain) or S~(Source).
If the sequence of the test pins is not 1, 2 and 3 in the reading direction, you can invert the sequence with the option
EBC\_STYLE=321 . The pin assignment is then shown with style ''321=...'', which will better match the usual
reading direction, if the testpin sequence is 3,2,1 .\\
Example: CFLAGS += EBC\_STYLE

  \item[NO\_NANO] specifies that the decimal prefix nano will not be used to display the measurement results.
So capacity values will be shown in \(\mu F\) instead of \(nF\).\\
Example: CFLAGS += NO\_NANO

  \item[NO\_LONG\_PINLAYOUT] can be set to prevent the long style of pin layout for graphical displays 
 like '' Pin  1=E 2=B 3=C''.
If the option is set, the short style is used instead like '' Pin  123=EBC''.\\
Example: CFLAGS += NO\_LONG\_PINLAYOUT

\item[PULLUP\_DISABLE] specifies, that you don't need the internal pull-up resistors.
 You must have installed a external pull-up resistor at pin~13 (PD7) to VCC, if you use this option.
This option prevents a possible influence of pull-up resistors at the measuring ports (Port~B and Port~C).\\
Example: CFLAGS += -DPULLUP\_DISABLE

  \item[ANZ\_MESS] this option specifies, how often an ADC value is read and accumulated.
You can select any value between 5 and 200 for building mean value of one ADC measurement.
Higher values result to better accuracy, but  longer measurement time.
One ADC measurement with 44~values takes about \(5ms\).\\
Example: CFLAGS += -DANZ\_MESS=25

  \item[POWER\_OFF] This option enables the automatic power off function. If you don't specify this option,
 measurements are done in a loop infinitely  until power is disconnected with a ON/OFF switch.
If you have the tester without the power off transistors, you can deselect the option POWER\_OFF.

If you have NOT selected the POWER\_OFF option with the transistors installed,
you can also shut down the tester, if you have selected the WITH\_MENU option.

You can also specify, after how many measurements without a founded part the tester will shut down.
The tester will also shut down the power after twice as much measurements are done in sequence without a
single failed part search. If you have forgotten to unconnect a test part, total discharging of battery is avoided. 
Specify the option with a form like CFLAGS += -DPOWER\_OFF=5 for a shut off after 5 consecutive measurements
without part found. Also 10~measurements with any founded part one after another will shut down.
Only if any sequence is interrupted by the other type, measurement continues.
The result of measurement stay on the display for 28~seconds for the single measurement, for the
multiple measurement version display time is reduced to 5~seconds (set in config.h).
If the start key is pressed a longer time on power on time, the display time is also 28~seconds for the multiple measurement.
The maximum value is 255 (CFLAGS += -DPOWER\_OFF=255).\\
Example 1: CFLAGS += -DPOWER\_OFF=5\\
Example 2: CFLAGS += -DPOWER\_OFF

  \item[BAT\_CHECK] enables the Battery Voltage Check. If you don't select this option, the version number of
software is output to the LCD instead.
This option is usefull for battery powered tester version to remember for the battery change.\\
Example: CFLAGS += -DBAT\_CHECK

  \item[BAT\_OUT] enables Battery Voltage Output on LCD (if BAT\_CHECK is selected).
 If your \(9V\) supply has a diode installed, use the BAT\_OUT=600 form to specify the threshold voltage (mV) of your diode
to adjust the output value.
Also the voltage loss of transistor T3 can be respected with this option.
 threshold level does not affect the voltage checking levels (BAT\_POOR).\\
Example 1: CFLAGS += -DBAT\_OUT=300\\
Example 2: CFLAGS += -DBAT\_OUT

  \item[BAT\_POOR] sets the poor level of battery voltage to the specified 1mV value.
The warning level of battery voltage is \(0.8V\) higher than the specified poor level, if the poor level is more than \(5.3V\).
If the poor level is \(5.3V\) or less, the warning level is \(0.4V\) higher. If the poor level is below \(3.25V\), the
warning level is only \(0.2V\) higher than the selected poor level and if the poor level is below \(1.3V\), the
warning level is only \(0.1V\) higher than the specified poor level.
Setting the poor level to low values such as \(5.4V\) is not recommended for rechargeable \(9V\) batteries,
because this increase the risk of battery damage by the reason of the deep discharge!
If you use a rechargeable \(9V\) Battery, it is recommended to use a Ready To Use type, because of the lower self-discharge.\\
Example for low drop regulator (\(5.4V\)): CFLAGS += -DBAT\_POOR=5400\\
Example for 7805 type regulator (\(6.4V\)): CFLAGS += -DBAT\_POOR=6400

  \item[DC\_PWR] This voltage level in mV units specify the battery voltage above which the tester
changes to the ''DC\_Pwr\_Mode''. Normally the tester operates in a battery mode, where all additional
functions are limited in time. With the ''DC\_Pwr\_Mode'' the tester runs the additional functions with unlimited time.
Because there is no DC-DC converter operating with \(0.9V\) input voltage,
the ''DC\_Pwr\_Mode'' is also entered, if the battery voltage is detected below \(0.9V\). \\
Example: CFLAGS += -DDC\_PWR=9500

 \item[BAT\_NUMERATOR] defines the numerator of a fraction used for scaling the input voltage to get the right
battery voltage.
For a normal voltage divider build with a \(10 k\Omega\) and a \(3.3 k\Omega\) resistor you get a fraction 
of (10000 + 3300)/3300. 
You should reduce the fraction to 133/33 .\\
Example: CFLAGS += -DBAT\_NUMERATOR=133

 \item[BAT\_DENOMINATOR] specifies the denominator of a fraction, which is used to scale the voltage value.\\
Example: CFLAGS += -DBAT\_DENOMINATOR=33

 \item[EXT\_NUMERATOR] defines the numerator of a fraction used for scaling the external input voltage.
 With a voltage divider build with a \(180 k\Omega\) and a \(20 k\Omega\) resistor the fraction is (180000 + 20000)/20000 .
You should reduce the fraction to 10/1 .\\
Example: CFLAGS += -DEXT\_NUMERATOR=10

 \item[EXT\_DENOMINATOR] specifies the denominator for the fraction to scale the external voltage. \\
Example: CFLAGS += -DEXT\_DENOMINATOR=1

  \item[INHIBIT\_SLEEP\_MODE] disable the use of the sleep mode of the processor.
Normaly the software uses for longer work breaks the sleep mode to avoid unneeded current consumption.
The usage of this sleep mode indeed spare battery capacity, but produce additional stress for the voltage regulator.\\
Example: INHIBIT\_SLEEP\_MODE = 1

  \item[PROGRAMMER] select your programmer type for avrdude interface program.
The correct selection of this option is needed, if you use the ``make upload'' or ``make fuses'' call
of this Makefile.
For further information please look to the manual pages of avrdude and online documentation~\cite{avrdude}.\\
Example: PROGRAMMER=avrisp2

  \item[BitClock] selects the Bit clock period for the Programmer. See the description of the -B parameter of avrdude.\\
Example: BitClock=5.0

  \item[PORT] select the port where avrdude can reach your microcontroller (atmega).
For further information please look to the manual pages of avrdude.\\
Example: PORT=usb

\end{description}

\begin{figure}[H]
  \begin{subfigure}[b]{8.6cm}
    \centering
    \includegraphics[width=8.3cm]{../PNG/AREF2_1V.png}
    \caption{from 5V to 1.1V }
    \label{pic:aref1}
  \end{subfigure}
  ~
  \begin{subfigure}[b]{8.6cm}
    \centering
    \includegraphics[width=8.3cm]{../PNG/AREF2VCC.png}
    \caption{from 1.1V to 5V}
    \label{pic:aref5}
  \end{subfigure}
  \caption{AREF switching with a \(1nF\) Capacitor}
\end{figure}


Additional parameters can be set in the files transistortester.h and config.h .
The file config.h contains global settings, defines the port / pin constellation,
 the clock frequency of the ADC and the resistor values used for measurement.
The file Transistortester.h contains the global variables and tables and also the text used for LCD output.
Normally there is no reason to change these values.
