\chapter{Configuring the TransistorTester}
\label{sec:config}
The complete software for the TransistorTester is available in source code.
The compilation of modules is controlled with a Makefile. The developement was done
at the Ubuntu Linux operating system with the GNU toolchain (gcc version 4.5.3).
It should be possible to use other Linux operating systems without problems.
To load the compiled data to the flash memory or
the EEprom memory, the tool avrdude (version 5.11svn) was taken by the Makefile, if you call ``make upload''.
 The program avrdude is available for Linux and Windows operating system.
The gnu C-compiler gcc is also taken by the AVR studio software
at the Windows operating system.
You can load the program data (.hex and .eep) also with other tools to the ATmega,
but only my Makefile version takes care to load the correct data to the choosed processor.
Avrdude loads only data to the ATmega if the Signature Bytes of the connected ATmega is
identical to the choosed one. 
If you alter the Makefile, all the software will be compiled new, if you call a ``make'' or
``make upload'' command. The software compiled for a ATmega8 does not run on a ATmega88.
The software compiled for a ATmega168 does not run on the ATmega88, even if the ATmega88 has enough flash memory! 
Be careful, if you don't use my Makefile.
With the correct options set, my software runs on the unchanged hardware of Markus F
(PARTNO=M8, NO option NO\_AREF\_CAP and NO PULLUP\_DISABLE option).
The clock rate can also be set to 8 MHz with fuses, no crystal is required!


The following options in the Makefile are avaiable to configure the software for your Tester.

\begin{description}
  \item[PARTNO] describes the target processor:\\
         m8 = ATmega8\\
         m88 or m88p = ATmega88\\
         m168 or m168p = ATmega168\\
         m328 or m328p = ATmega328\\
    example:  PARTNO = m168
  \item[UI\_LANGUAGE] specifies the favored Language\\
    LANG\_ENGLISH, LANG\_GERMAN, LANG\_POLISH, LANG\_CZECH, LANG\_SLOVAK and LANG\_SLOVENE is currently avaiable \\
    example:  UI\_LANGUAGE = LANG\_ENGLISH
  \item[LCD\_CYRILLIC] is only needed for a LCD-display with cyrillic character set. The \(\mu\) and \(\Omega\) character
is not avaiable with the cyrillic character set.
If you specify this option, both characters are loaded to the LCD with software.\\
example: CFLAGS += -DLCD\_CYRILLIC
  \item[WITH\_SELFTEST] If you specify this Option, software will include a selftest function.
Selftest will be started, if you connect all three probes together and start measurement.\\
example: CFLAGS += -DWITH\_SELFTEST
  \item[AUTO\_CAL] The zero offset for capacity measurement and the port output resistor values will be written additionally
to the EEprom with the selftest routine. Additionally the offset voltage of the analog comparator (REF\_C\_KORR) and the
voltage offset of the internal reference voltage (REF\_R\_KORR) will be measured automatically, if you connect a
capacitor with a capacity value between \(100 nF\) and \(20 \mu F\) to pin~1 and pin~3 after measurement of capacity zero offset. 
All found values will be written to EEprom and will be used for further measurements automatically.\\
example: CFLAGS += -DAUTO\_CAL
  \item[FREQUENCY\_50HZ] At the end of selftest a 50~Hz Signal will be generated on Port~2 and Port~3 for up to one minute.\\
example: CFLAGS += -DFREQUENCY\_50HZ
  \item[R\_MESS] enables the resistor measurement. This option should allways be set.\\
example: CFLAGS += -DR\_MESS
  \item[C\_MESS] enables the capacity measurement. This option should allways be set.\\
example: CFLAGS += -DC\_MESS
  \item[CAP\_EMPTY\_LEVEL]  This option defines the voltage level for discharged capacitor (mV units).
You can set the level to higher value as 3mV, if the tester does not finish discharging of capacitors.
In this case the tester ends after longer time with the message ``Cell!''.\\
example: CFLAGS += -DCAP\_EMPTY\_LEVEL=3
  \item[WITH\_AUTO\_REF] specifies, that reference voltage is read to get the actual factor for capacity measuring of low capacity values (below \(40\mu F\)).\\
example:  CFLAGS += -DWITH\_AUTO\_REF
  \item[REF\_C\_KORR] specifies a offset for readed reference voltage in mV units.
This can be used to adjust the capacity measurement of little capacitors.
A correction value of 10 results to about 1~percent lower measurement results.
If the option AUTO\_CAL is selected together with the WITH\_SELFTEST option, the REF\_C\_KORR will be
a offset to the measured voltage difference of the test capacitor and the internal reference voltage.\\
example:  CFLAGS += -DREF\_C\_KORR=14
  \item[C\_H\_KORR] specifies a correction value for the measurement of big capacitor values.
A value of 10 results to 1~percent lower measurement results.\\
example:  CFLAGS += -DC\_H\_KORR=10
  \item[AUTOSCALE\_ADC] enables the automatic scale switchover of the ADC to either VCC or internal reference.
Internal reference gives a 2.56V scale for ATmega8 and a 1.1V scale for other processors.\\
example: CFLAGS += -DAUTOSCALE\_ADC
  \item[NO\_AREF\_CAP] tells your Software, that you have no capacitor (\(100 nF\)) installed at pin AREF (pin 21).
This enables a shorter wait-time for the AUTOSCALE\_ADC scale switching of the ADC.
A \(1 nF\) capacitor was tested in this mode without detected errors.
Figure \ref{pic:aref1} and \ref{pic:aref5} show the switching time with a \(1 nF\) capacitor.
As you can see the switching from 5V to 1.1V is much slower than switching back to 5V. If you
have still installed the \(100 nF\), switching time will be about factor 100 longer!\\
example: CFLAGS += -DNO\_AREF\_CAP
  \item[REF\_R\_KORR] specifies a offset for the internal ADC-reference voltage in mV units.
With this offset a difference by switching from VCC based ADC reference to internal ADC reference for resistor measurement can be adjusted.
If you select the AUTO\_CAL option of the selftest section, this value is only a additionally offset to the found voltage 
difference in the AUTO\_CAL function.\\
example: CFLAGS += -DREF\_R\_KORR=10
  \item[OP\_MHZ] tells your software at which Clock Frequency in MHz your Tester will operate.
The software is tested only for 1 MHz, 8MHz and additionally 16MHz. 
The 8MHz operation is recommended for better resolution of capacity and inductance measurement.\\
example: OP\_MHZ = 8
  \item[USE\_EEPROM] specifies if you wish to locate fix text and tables in EEprom Memory. Otherwise the flash memory is used.
Recommended is to use the EEprom (option set).\\
example: CFLAGS += -DUSE\_EEPROM
  \item[PULLUP\_DISABLE] specifies, that you don't need the internal pull-up resistors.
 You must have installed a external pull-up resistor at pin 13 (PD7) to VCC, if you use this option.
This option prevents a possible influence of pull-up resistors at the measuring ports (Port B and Port C).\\
example: CFLAGS += -DPULLUP\_DISABLE
  \item[ANZ\_MESS] this option specifies, how often an ADC value is read and accumulated.
You can select any value between 5 and 200 for building mean value of one ADC measurement.
Higher values result to better accuracy, but  longer measurement time.
One ADC measurement with 44~values takes about 5ms.\\
example: CFLAGS += -DANZ\_MESS=25
  \item[POWER\_OFF] This option enables the automatic power off function. If you don't specify this option,
 measurements are done in a loop infinitely  until power is disconnected with a ON/OFF switch.
If you have the tester without the power off transistors, you can deselect the option POWER\_OFF.
If you have NOT selected the POWER\_OFF option with the transistors installed,
you can stop measuring by holding the key several seconds after a result is displayed until the time out message is shown.
After releasing the key, the tester will be shut off by timeout.
You can also specify, after how many measurements without a founded part the tester will shut down.
The tester will also shut down the power after twice as much measurements are done in sequence without a
single failed part search. If you have forgotten to unconnect a test part, total discharging of battery is avoided. 
Specify the option with a form like CFLAGS += -DPOWER\_OFF=5 for a shut off after 5 consecutive measurements
without part found. Also 10~measurements with any founded part one after another will shut down.
Only if any sequence is interrupted by the other type, measurement continues.
The result of measurement stay on the display for 10~seconds for the single measurement, for the
multiple measurement version display time is reduced to 3~seconds (set in config.h).
If the start key is pressed a longer time on power on time, the display time is also 10~seconds for the multiple measurement.
The maximum value is 255 (CFLAGS += -DPOWER\_OFF=255).\\
example 1: CFLAGS += -DPOWER\_OFF=5\\
example 2: CFLAGS += -DPOWER\_OFF
  \item[BAT\_CHECK] enables the Battery Voltage Check. If you don't select this option, the version number of
software is output to the LCD instead.
This option is usefull for battery powered tester version to remember for the battery change.\\
example: CFLAGS += -DBAT\_CHECK
  \item[BAT\_OUT] enables Battery Voltage Output on LCD (if BAT\_CHECK is selected).
 If your 9V supply has a diode installed, use the BAT\_OUT=600 form to specify the threshold voltage (mV) of your diode
to adjust the output value.
Also the voltage loss of transistor T3 can be respected with this option.
 threshold level does not affect the voltage checking levels (BAT\_POOR).\\
example 1: CFLAGS += -DBAT\_OUT=300\\
example 2: CFLAGS += -DBAT\_OUT
  \item[BAT\_POOR] sets the poor level of battery voltage to the specified 100mV (1/10 Volt) value.
The warning level of battery voltage is always 1V higher than the specified poor level.
Setting the poor level to low values such as 5.4V is not recommended for rechargeable 9V batteries,
because this increase the risk of battery damage by the reason of the deep discharge!
If you use a rechargeable 9V Battery, it is recommended to use a Ready To Use type, because of the lower self-discharge.\\
example for low drop regulator (5.4V): CFLAGS += -DBAT\_POOR=54\\
example for 7805 type regulator (6.4V): CFLAGS += -DBAT\_POOR=64
  \item[PROGRAMMER] select your programmer type for avrdude interface program.
The correct selection of this option is needed, if you use the ``make upload'' or ``make fuses'' call
of this Makefile.
For further information please look to the manual pages of avrdude and online documentation~\cite{avrdude}.\\
example: PROGRAMMER=avrisp2
  \item[PORT] select the port where avrdude can reach your microcontroller (atmega).
For further information please look to the manual pages of avrdude.\\
example: PORT=usb

\end{description}

\begin{figure}[H]
  \begin{subfigure}[b]{9cm}
    \centering
    \includegraphics[width=9cm]{../PNG/AREF2_1V.png}
    \caption{from 5V to 1.1V }
    \label{pic:aref1}
  \end{subfigure}
  ~
  \begin{subfigure}[b]{9cm}
    \centering
    \includegraphics[width=9cm]{../PNG/AREF2VCC.png}
    \caption{from 1.1V to 5V}
    \label{pic:aref5}
  \end{subfigure}
  \caption{AREF switching with a \(1nF\) Capacitor}
\end{figure}


Additional parameters can be set in the files transistortester.h and config.h .
The file transistortester.h contains global variables and defines the port / pin constellation
and the resistor values used for measurement.
The file config.h specifies parameter for different processor types, wait times and the clock
frequency of the ADC. Normally there is no reason to change these values.
